\chapter{System Model}
\label{chap:sysmod}

Following some calculus:

\section{Motor Orientation}
We assume perfect spherical shape. That reduces the degree of freedom for the motor placement from 6 to 3. We therefore describe the motor placement by a rotation about a arbitrary axis $\mathbf{n}$
\begin{equation}
\mathbf{n} = \left(n_x, n_y, n_z\right)^\top
\end{equation}
and an rotation angle $\varphi$.
With the skew symmetric cross matrix
\begin{equation}
\textbf{a}^\times = \left[
\begin{array}{ccc}
0 & -a_3 & a_2 \\
a_3 & 0 & -a_1 \\
-a_2 & a_1 & 0
\end{array} \right]
\end{equation}
we can represent this rotation with the Gibbs-Rodriguez parameters as
\begin{equation}
\boldsymbol{\lambda} = \mathbf{n}\tan{\left(\frac{\varphi}{2}\right)}
\end{equation}
Whereas this is a minimal representation (which always include singularity - here at $\varphi=\pi / 2$) we can also use non-minimal representation to overcome singularities. We therefore use the quaternion, as initially suggested by Hamilton.
\begin{equation}
\mathbf{q} = \left( \begin{array}{c} 
q_0 \\ \mathbf{q}_{1:3} 
\end{array} \right) \qquad
q_0 = \cos\left(\frac{\varphi}{2}\right) \qquad
\mathbf{q}_{1:3} = \mathbf{n}\sin\left(\frac{\varphi}{2}\right)
\end{equation}

\subsection{Rotation Matrix (or Direction Cosine Matrix)}
For Gibbs-Rodriguez parameters
\begin{equation}
\mathbf{C} = \mathbf{I} + \frac{2}{1+\boldsymbol{\lambda}^\top \boldsymbol{\lambda}}
\left(\boldsymbol{\lambda}^\times + \left(\boldsymbol{\lambda}^\times\right)^2\right)
\end{equation}
and for quaternions
\begin{equation}
\mathbf{C} = \mathbf{I} + 2 q_0 \mathbf{q}_{1:3}^\times + 2 \left(\mathbf{q}_{1:3}^\times\right)^2
\end{equation}
In some cases (e.g. gradient based optimization) the Jacobian might be necessary. The Jacobian of the Rotation Matrix must therefore be derived with respect to the parameters. For the quaternion parametrization the relation is quadratic, therefore the Jacobians are linear.
\begin{equation}
\frac{\partial \mathbf{C}}{\partial q_0} = 2 \left[ 
\begin{array}{ccc}
 q_0 & -q_3 &  q_2 \\
 q_3 &  q_0 & -q_1 \\
-q_2 & -q_1 &  q_0
\end{array} \right]
\end{equation}
\begin{equation}
\frac{\partial \mathbf{C}}{\partial q_1} = 2 \left[ 
\begin{array}{ccc}
q_1 & q_2 &   q_3 \\
q_2 & -q_1 & -q_0 \\
q_3 & q_0 &  -q_1
\end{array} \right]
\end{equation}
...

It is also possible to get an algebraic expression for the Jacobians using Gibbs-Rodriguez parameters and use it for gradient based optimization algorithms.
\begin{equation}
\frac{\partial \mathbf{C}}{\partial \lambda_3} = \left(\begin{array}{ccc} \frac{4\, {\lambda_3}\, \left({{\lambda_2}}^2 + {{\lambda_3}}^2\right)}{{\sigma}^2} - \frac{4\, {\lambda_3}}{\sigma} & \frac{4\, {\lambda_3}\, \left({\lambda_3} - {\lambda_1}\, {\lambda_2}\right)}{{\sigma}^2} - \frac{2}{\sigma} & \frac{2\, {\lambda_1}}{\sigma} - \frac{4\, {\lambda_3}\, \left({\lambda_2} + {\lambda_1}\, {\lambda_3}\right)}{{\sigma}^2}\\ \frac{2}{\sigma} - \frac{4\, {\lambda_3}\, \left({\lambda_3} + {\lambda_1}\, {\lambda_2}\right)}{{\sigma}^2} & \frac{4\, {\lambda_3}\, \left({{\lambda_1}}^2 + {{\lambda_3}}^2\right)}{{\sigma}^2} - \frac{4\, {\lambda_3}}{\sigma} & \frac{2\, {\lambda_2}}{\sigma} + \frac{4\, {\lambda_3}\, \left({\lambda_1} - {\lambda_2}\, {\lambda_3}\right)}{{\sigma}^2}\\ \frac{2\, {\lambda_1}}{\sigma} + \frac{4\, {\lambda_3}\, \left({\lambda_2} - {\lambda_1}\, {\lambda_3}\right)}{{\sigma}^2} & \frac{2\, {\lambda_2}}{\sigma} - \frac{4\, {\lambda_3}\, \left({\lambda_1} + {\lambda_2}\, {\lambda_3}\right)}{{\sigma}^2} & \frac{4\, {\lambda_3}\, \left({{\lambda_1}}^2 + {{\lambda_2}}^2\right)}{{\sigma}^2} \end{array}\right)
\end{equation}
where
\begin{equation}
\sigma = {\lambda_1}^2 + {\lambda_2}^2 + {\lambda_3}^2 + 1
\end{equation}
\subsection{Input-Output dynamics}
The dynamics of the blimp system is given by its differential equations in the world or blimp coordinates respectively:
\begin{equation}
\label{eqn_dyn_angular_velocity}
\dot{\boldsymbol{\omega}}_b^b = \mathbf{J}_b^{-1} \left( \mathbf{M}_b  - \boldsymbol{\omega}_b^b \times \mathbf{J}_b \boldsymbol{\omega}_b^b \right)
\end{equation}
\begin{equation}
\label{eqn_dyn_orientation}
\dot{\mathbf{q}}_w^{w,b} = \frac{1}{2} \mathbf{q}_w^{w,b} \otimes \left[
\begin{array}{c}
	0 \\ \boldsymbol{\omega}_b^b
\end{array} \right]
\end{equation}
\begin{equation}
\label{eqn_dyn_velocity}
\dot{\mathbf{v}}_w^b = \mathbf{C}_{w,b}\mathbf{\mathcal{M}}_b^{-1} \mathbf{F}_b
\end{equation}
\begin{equation}
\label{eqn_dyn_position}
\dot{\mathbf{p}}_w^{w,b} = \mathbf{v}_w^b
\end{equation}

The gyro measures the angular velocity in the IMU frame. It can be modelled as
\begin{equation}
\mathbf{y}_{g} = \mathbf{C}_{i,b} \boldsymbol{\omega}_b^b + \mathbf{w}_g
\end{equation}
where $\mathbf{w}_g$ denotes the gyro measurement noise (gaussian.. mean and variance..).

\subsection{Optimization Problem}
One approach to state the minimization problem is to use the sum-of-squares of the difference between the measured and the estimated angular acceleration of the blimp as in equation \eqref{eqn_minimize_problem_alpha}. As described below, the estimated angular velocity is dependent on the gyro measurements $\mathbf{y}_g$ as a consequence of the body dynamics as stated in \eqref{eqn_dyn_angular_velocity}.

\begin{equation}
\label{eqn_minimize_problem_alpha}
S = \sum_{j=1}^n {\mathbf{r}_j^\top \mathbf{r}_j} = \sum_{j=1}^n {
\left( \frac{\mathbf{C}_{i,b}^{-1} \Delta \mathbf{y}_g}{\Delta t} - \hat{\boldsymbol{\alpha}}_b^b \right)_j^\top
\left( \frac{\mathbf{C}_{i,b}^{-1} \Delta \mathbf{y}_g}{\Delta t} - \hat{\boldsymbol{\alpha}}_b^b \right)_j }
\end{equation}

where $j$ indicates the time stamps. The estimated angular velocity of the blimp is
\begin{equation}
\label{eqn_est_alpha_1}
\hat{\boldsymbol{\alpha}}_b^b = \hat{\mathbf{J}}_b^{-1} \left( \hat{\mathbf{M}}_b  - \hat{\boldsymbol{\omega}}_b^b \times \hat{\mathbf{J}}_b \hat{\boldsymbol{\omega}}_b^b \right)
\end{equation}
In a first approach, we assume that we know the inertia tensor $\mathbf{J}$. For a second approach, this should be depending on the estimated motor positions. The estimated angular velocity can be approached by the gyro measurement, transformed into blimp coordinates. Then we get
\begin{equation}
\label{eqn_est_alpha_2}
\hat{\boldsymbol{\alpha}}_b^b = \mathbf{J}_b^{-1} \left( \hat{\mathbf{M}}_b  - \left( \mathbf{C}_{i,b}^{-1} \mathbf{y}_g \right) \times \left( \mathbf{J}_b \mathbf{C}_{i,b}^{-1} \mathbf{y}_g \right) \right)
\end{equation}

Inserting \eqref{eqn_est_alpha_2} into the objective function \eqref{eqn_minimize_problem_alpha} yields

\begin{equation}
\label{eqn_objective_short}
\begin{aligned}
\mathbf{r}_j &= \left(
\underbrace{
\frac{\mathbf{C}_{i,b}^{-1} \Delta \mathbf{y}_g}{\Delta t} + \mathbf{J}_b^{-1} \left( \left( \mathbf{C}_{i,b}^{-1} \mathbf{y}_g \right) \times \left(\mathbf{J}_b \mathbf{C}_{i,b}^{-1} \mathbf{y}_g \right) \right)}_\mathcal{Y}
 - \mathbf{J}_b^{-1} \hat{\mathbf{M}}_b \right)_j \\
 &= \left( \mathcal{Y} - \hat{r} \mathbf{J}_b^{-1} \hat{\mathcal{C}}_{b,m} \mathbf{u} \right)_j
\end{aligned}
\end{equation}

where estimated moment acting on the blimp $\hat{\mathbf{M}}_b$ has been substituted by
\begin{equation}
\label{eqn_est_moment_reduction}
\begin{aligned}
\hat{\mathbf{M}}_b &= \sum_{k=1}^N { \hat{\mathbf{C}}_{b,m_k} \left(
\left[ \begin{array}{c}
0 \\ 0 \\ - \hat{r}
\end{array} \right]
\times 
\left[ \begin{array}{c}
F^{m_k}_x \\ F^{m_k}_y \\ 0
\end{array} \right]
\right)} \\
&= \hat{r} \cdot \underbrace{ \left[
\begin{array}{ccc}
\hat{\mathbf{C}}_{b,m_1} & \cdots \hat{\mathbf{C}}_{b,m_N}
\end{array}
\right] }_{\hat{\mathcal{C}}_{b,m}}
\underbrace{ \left[
\begin{array}{ccccccc}
 F^{m_1}_y & - F^{m_1}_x & 0 & \cdots &F^{m_N}_y & -  F^{m_N}_x & 0 
\end{array}
\right]^\top }_{\mathbf{u}}
\end{aligned}
\end{equation}

\begin{equation}
\hat{\mathbf{M}}_{b,j} = \hat{r} \cdot \hat{\mathcal{C}}_{b,m} \mathbf{u}_j
\end{equation}

Now we stack the time steps to a single vector. This can be done as follows with the stacked measurements $\mathcal{Y}$ and the stacked inputs $\mathcal{U}$

\begin{equation}
\label{eqn_obj_stacked}
\mathbf{r} = \mathcal{Y} - \hat{r} \left[ \begin{array}{ccc}
\mathbf{J}_b^{-1} \hat{\mathcal{C}}_{b,m} & & \\ 
& \ddots & \\ 
& & \mathbf{J}_b^{-1} \hat{\mathcal{C}}_{b,m}
\end{array} \right]
\mathcal{U}
\end{equation}
\begin{equation}
\mathcal{Y} = \left[ \begin{array}{ccc}
\mathcal{Y}_1^\top & \cdots & \mathcal{Y}_n^\top
\end{array} \right]^\top
\end{equation}
\begin{equation}
\mathcal{U} = \left[ \begin{array}{ccc}
\mathbf{u}_1^\top & \cdots & \mathbf{u}_n^\top
\end{array} \right]^\top
\end{equation}

%\begin{equation}
%\hat{\mathcal{M}}_b = \hat{r}\cdot \left[ \begin{array}{ccc}
%\hat{\mathcal{C}}_{b,m} & & \\ & \ddots & \\ & & \hat{\mathcal{C}}_{b,m}
%\end{array} \right]
%\mathcal{U}
%\end{equation}
%Finally, we can rewrite the objective function for the minimization problem as
%\begin{equation}
%S = \sum_j{ \left( \frac{\mathbf{C}_{i,b}^{-1} \Delta \mathbf{y}_g}{\Delta t} - \hat{\boldsymbol{\alpha}}_b^b \right)_j^2 }
%= \left(\hat{\mathcal{M}}_b - \right)^\top 
%  \left(\hat{\mathcal{M}}_b - \right)
%\end{equation}

So far, no assumption concerning the optimization parameters has been made. The optimization parameters are $\hat{r}$ and $\hat{\mathcal{C}}_{b,m}$. These are $p\cdot N+1$ parameters, where $N$ is the number of motors and $p = {3,4}$ the number of parameters for the rotation. It is also possible to split the problem into single problems, where always only one motor is active at the same time. Then the number of parameters drops to $p+1$.

\subsection{Optimization Problem (parameterized)}
If we use quaternion parametrization, we get
\begin{equation}
\mathbf{J}_b^{-1}\mathbf{M}_b = 
\left[\begin{array}{c}
-G_{11} \sigma_1(\xi,\theta) + G_{12} \sigma_2(\xi,\theta) - G_{13} \sigma_3(\xi,\theta) \\
-G_{21} \sigma_1(\xi,\theta) + G_{22} \sigma_2(\xi,\theta) - G_{23} \sigma_3(\xi,\theta) \\
-G_{31} \sigma_1(\xi,\theta) + G_{32} \sigma_2(\xi,\theta) - G_{33} \sigma_3(\xi,\theta) \\
\end{array}\right]
\end{equation}
where
\begin{align*}
\sigma_1(\xi,\theta) &= - r\, {\xi_1}\, \left(2\, {\theta_1}\, {\theta_4} - 2\, {\theta_2}\, {\theta_3}\right) + r\, {\xi_2}\, \left(2\, {{\theta_3}}^2 + 2\, {{\theta_4}}^2 - 1\right) \\
\sigma_2(\xi,\theta) &= r\, {\xi_1}\, \left(2\, {{\theta_2}}^2 + 2\, {{\theta_4}}^2 - 1\right) + r\, {\xi_2}\, \left(2\, {\theta_1}\, {\theta_4} + 2\, {\theta_2}\, {\theta_3}\right) \\
\sigma_3(\xi,\theta) &= r\, {\xi_1}\, \left(2\, {\theta_1}\, {\theta_2} + 2\, {\theta_3}\, {\theta_4}\right) + r\, {\xi_2}\, \left(2\, {\theta_1}\, {\theta_3} - 2\, {\theta_2}\, {\theta_4}\right)
\end{align*}

\subsection{Observability}
Lets state the system equations again. Note it is a \textit{control-affine} system (or \textit{input-linear}. \\
State equations
\begin{equation}
\dot{\mathbf{x}} = 
\left[ \begin{array}{c}
\dot{\boldsymbol{\omega}} \\
\dot{\mathbf{q}} \\
\dot{\mathbf{J}}
\end{array} \right]
= \mathbf{f}(\mathbf{x}, \mathbf{u}) =
\left[ \begin{array}{c}
r \mathbf{J}^{-1} \mathbf{C}(\mathbf{q}) \mathbf{u} - \mathbf{J}^{-1}(\boldsymbol{\omega} \times \mathbf{J} \boldsymbol{\omega} ) \\
0 \\
0
\end{array} \right]
\end{equation}
Measurement equations
\begin{equation}
\mathbf{y} = \mathbf{h}(x) = \boldsymbol{\omega}
\end{equation}

Now some hints about differentiating:
\begin{align}
\frac{\partial}{\partial \mathbf{a}} \left( \mathbf{a} \times \mathbf{b} \right)
& = - \mathbf{b} ^\times \\
\frac{\partial}{\partial \mathbf{a}} \left( \mathbf{b} \times \mathbf{a} \right)
& = \mathbf{b} ^\times
\end{align}
We will especially use
\begin{align}
\frac{\partial}{\partial \mathbf{a}} \left( \mathbf{a} \times \textbf{M} \mathbf{a} \right)
& = \mathbf{a}^\times \mathbf{M} - (\mathbf{M} \mathbf{a})^\times
\end{align}
