\chapter{Problem Evaluation}
\label{chap:problem_evaluation}

The main goal of this work is the \textit{Estimation of Actuation Configuration for a Multi-Actuated Blimp}.
The need for a good estimate of the actuation configuration is mainly given by the calculation of the allocation part for the controller.
Additional value can be achieved by estimating further properties, e.g. the offset between COG and COB.
If this offset is known, it is possible to calculate the required taring weight apportionment to get any desired COG offset\footnote{
In the simulation (see \cref{sec:simulation}), such an algorithm is already implemented to place the taring weights.
}.
\\

There exist various methods within system identification.
% DOES SYSTEM IDENTIFICATION INCLUDE ONLINE EXTIMATION?
Online estimation techniques are preferred for dynamically changing properties (state estimation).
Offline estimation is preferred for estimating constant properties (parameter estimation).
The actuation configuration is assumed to be constant.
Therefore we will use offline estimation in this work.
Nevertheless, a combined (online) estimator for parameters and states would be interesting but goes beyond the scope of this thesis. \\
Offline estimation means that we collect a batch of data and calculate the estimate of the actuation configuration out of this data.
If the parameters change slowly over time, it is also possible to repeat the batch optimization after some time to consider these changes.
\\

For the offline estimation of the actuation configurations, we assume the system states to be known.
That means we use the available information about the states from the state estimator which is already implemented on the system.\\
For bulky, but rotation symmetric systems as blimps, aerodynamic effects have much more influence on the translational movements than on the rotational movements.
Effects like flow separation, drag crisis or the inertia of the surrounding fluid mainly influence the translational movements.
On rotational movements, aerodynamic effects have a much smaller impact.
Since it is a difficult task to model aerodynamic effects and this would lead to many more unknown parameters, we decided to restrict ourself on the investigation of the rotational movement.
We will see below that this approach is greatly capable of estimating the actuation configuration.
Nevertheless, we designed our estimator such that it would be possible to extend it by considering translational movements too.

%...Title is \textsc{Estimation of Actuation Configuration for a Multi-Actuated Blimp}
%...inside System Identification methods, why did we pick batch optimization
%...why did we restrict to Gyro / angular acceleration