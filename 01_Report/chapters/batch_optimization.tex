\chapter{Batch Optimization}
\label{chap:batch}

Figure \ref{fig:batch_opti} gives an overview on how the batch optimization problem is solved.
On one hand, the real system is fed with some specific inputs and the system behaviour is measured by the available sensors.
On the other hand, virtual sensor measurements are calculated for the same inputs using a parameterized system model.
Our goal is to minimize the difference between the real and virtual system output.
For that, we formulate a cost function

\begin{equation}
S(\theta) = \sum_{j=1}^n ( y_j - f_j(x_j, \theta) )^2
\end{equation}

where $y_j$ are the measurements and $f_j(x_j, \theta)$ are the corresponding model outputs at time step $j$ and look for the parameters $\theta$ that minimize this function.
\\

In the following, we will describe the system model and its parameters.
Then we will show how we solved the minimization problem and what results we can expect.

\begin{figure}[btp]
\label{fig:batch_opti}
\centering
\includegraphics[width=0.85\textwidth]{images/problem_formulation.pdf}
\caption{Scheme for optimization of parameterized system model.}
\end{figure}

\section{System Model}
\label{sec:system_model}
A system model for Skye has been stated by Weichart in \cite{Weichart2012} and is summarized in table \ref{tab:sys_mod}.

\begin{table}[htb!]
\centering
\label{tab:sys_mod}
$\begin{array}{ll}
\toprule
\dot{\boldsymbol{\omega}}_b^b &= \mathbf{J}_b^{-1} \left( \mathbf{M}_b  - \boldsymbol{\omega}_b^b \times \mathbf{J}_b \boldsymbol{\omega}_b^b \right) \\

\dot{\mathbf{q}}_w^{b,w} &= \frac{1}{2} \mathbf{q}_w^{b,w} \otimes \left[
\begin{array}{c}
	0 \\ \boldsymbol{\omega}_b^b
\end{array} \right] \\

\dot{\mathbf{v}}_b^b &= \mathbf{\mathcal{M}}_b^{-1} \mathbf{F}_b \\

\dot{\mathbf{p}}_w^{b,w} &= \mathbf{v}_w^b \\

\bottomrule
\end{array}$
\caption{Equations of motion for Skye}
\end{table}

It consists on the differential equations for the angular velocity $\boldsymbol{\omega}$, orientation quaternion $\mathbf{q}$, velocity $\mathbf{v}$, and position $\mathbf{p}$.
Skye is assumed as one rigid body.
Active forces and moments with respect to its center of gravity are concentrated as $\mathbf{F}$ and $\mathbf{M}$ respectively.
The total mass $\mathbf{\mathcal{M}}$ includes the rigid mass, helium mass as well as the virtual mass compensating for the inertia of the surrounding fluid.
The moment of inertia (or inertia tensor) $\mathbf{J}$ does not contain a virtual part as long as we focus on sphere like blimps.
\\
For the batch optimization, we content ourself with the angular acceleration $\boldsymbol{\alpha}$ as outlined in chapter \ref{chap:design_evaluation} and will have a closer look at the corresponding equation now.

\begin{equation}
\label{eq:angular_accel}
\boldsymbol{\alpha}_b^b = \mathbf{J}_b^{-1} \left( \mathbf{M}_b  - \boldsymbol{\omega}_b^b \times \mathbf{J}_b \boldsymbol{\omega}_b^b \right)
\end{equation}

The change of angular velocity is given by the active moments $\mathbf{M}$ as well as the nutation term $\boldsymbol{\omega} \times \mathbf{J} \boldsymbol{\omega}$.
\\
The latter term influences the rotation axis such that the free body will asymptotically end in a rotation around its smallest or or largest principle inertia axis.
For a homogeneous sphere, the inertia tensor is diagonal with all its diagonal elements equal and the cross product in \eqref{eq:angular_accel} is therefore zero (because $\boldsymbol{\omega}$ and $\mathbf{J}\boldsymbol{\omega}$ are parallel).
As Skye itself is not as perfect symmetric than a sphere (compare section \ref{sub:par_inertia}) the nutation term will be considered although it is much smaller than the active moment.
\\
As shown in \eqref{eq:moments}, the active moment consists of an actuation term $\mathbf{M}^{actuation}$, a gravitation term $\mathbf{M}^{gravity}$, and an aerodynamic term $\mathbf{M}^{aero}$.

\begin{equation}
\label{eq:moments}
\mathbf{M}_b = \underbrace{\sum_{k=1}^N  \left[  \mathbf{C}_{b,m_k} \left( \mathbf{p}^{m_k,cog}_{m_k} \times \mathbf{F}^k_{m_k} \right)  \right]}_{\mathbf{M}^{actuation}}
-
\underbrace{
 \left( \mathbf{p}^{cob,cog}_b \times (\mathbf{C}_{b,w}m\mathbf{g}_w) \right)
}_{\mathbf{M}^{gravity}}
+
\mathbf{M}^{aero}
\end{equation}

For the actuation term, the moment
(cross product between position vector $\mathbf{p}^{cog,m_k}$ from COG to thruster's point of action and motor force $\mathbf{F}^k$)
is first calculated in the local motor coordinate frames $m_k$ and then transformed by the rotation matrix $\mathbf{C}_{b,m_k}$ to blimp coordinates $b$ for each motor $k=1,...,N$.

The gravity term includes the offset $\mathbf{p}^{cob,cog}$ from COG to COB and 
the gravitation force $m\mathbf{g}$ which is transformed by the rotation matrix $\mathbf{C}_{b,w}$ from world coordinates $w$ to blimp coordinates $b$.

The aerodynamic term can be modelled as aerodynamic friction. According to Kundu, the wall shear stress is often expressed in terms of the dimensionless skin friction coefficient $c_f$ as \citet{Kundu2012}
\begin{equation*}
\tau_{wall} = 
\frac{1}{2} c_f \rho_{air} v^2 .
\end{equation*}

If only rotational movement of the blimp is considered, integrating the wall shear stress over the sphere leads to a moment counter acting the rotation with magnitude
\begin{align*}
\| \mathbf{M}^{aero} \|
&= \int_A \xi \tau_{wall} dA \\
&= \frac{1}{2} c_f \rho_{air} \int_A \xi (\xi \omega)^2 dA \\
&= \frac{1}{2} c_f \rho_{air} \omega^2 \int_{-r}^{r} (r^2-z^2)^{\frac{3}{2}} \sqrt{r^2-z^2} 2\pi dz \\
&= \frac{1}{2} c_f \rho_{air} \omega^2 \frac{32}{15} \pi r^5
\end{align*}

Nevertheless, this does not consider the drag of all components which are attached on the hull (e.g. actuation units or handles) which will have a large effect on the aerodynamics.
Measurements showed that the aerodynamic drag is more than a magnitude smaller than the usual actuation moments ($c_f \approx 0.02; \|\mathbf{M}^{aero}(\omega = \unit[0.8]{rad/s})\| \approx \unit[0.2]{Nm}$). Therefore we neglect this term and do not further bother about its direction.

%The position vector $\mathbf{p}^{m_k,cog}_{m_k}$ is expressed in motor coordinates too.
%As shown in section \ref{sub:par_position}, this simplifies the equation a lot if the motors are placed on a sphere. The motors forces $\mathbf{F}_{m_k}$ have only nonzero components in x and y direction (see figure \ref{fig:frames})
%
%\begin{equation}
%\mathbf{F}_{m_k} = \left[ \begin{array}{c}
%F_k\cos(\varphi_k) \\
%F_k\sin(\varphi_k) \\
%0
%\end{array} \right]
%\end{equation}
%
%and the thrust $F_k$ and angle $\varphi_k$ are accurately known for the selected data samples (compare section \ref{sub:data_selection}).

\section{Parameterization}
For the angular acceleration model introduced above, we formulate the parameterized system model as

\begin{eqnarray}
\label{eq:sys_mod}
\lefteqn{
\mathbf{f}(\mathbf{x}, \boldsymbol{\theta}) = {}} \\
& & {} \mathbf{J}_b^{-1} \left( 
\sum_{k=1}^N  \left[  \mathbf{C}_{b,m_k} \left( \mathbf{p}^{m_k,cog}_{m_k} \times \mathbf{F}^k_{m_k} \right)  \right]
-
\left( \mathbf{p}^{cob,cog}_b \times (\mathbf{C}_{b,w}m\mathbf{g}_w) \right)
- \boldsymbol{\omega}_b^b \times \mathbf{J}_b \boldsymbol{\omega}_b^b \right) \nonumber
\end{eqnarray}

Every variable in \eqref{eq:sys_mod} must either be measurable, known or a parameter
\footnote{Here we use only constants for known and parameterized variables. For time dependent variables, e.g. spline functions can be used CITE FURGALE.}.
In order to find a unique solution for the parameter set, the problem needs to be observable.


As introduced in section \ref{}... we want ... and we must ... because they are unsure/hard to measure
\subsection{Observability}
\label{sub:observability}
... tell what is (is not) observable. \\ Maybe also show why. \\ Cite \cite{hermann1977}

\subsection{Orientation}
\label{sub:par_orientation}
... Special Orthogonal Group
... Rodriquez-Gibbs parameters

\subsection{Position}
\label{sub:par_position}

\subsection{Inertia Tensor}
\label{sub:par_inertia}

\section{Nonlinear Least Squares}
... Formulate minimization problem (unconstraint) as least-squares.
... Cite \cite{Seber} (Nonlinear Regression) and maybe also \cite{Draper} (Applied Regression Analysis)

\section{Levenberg-Marquardt Algorithm}
... Again cite \cite{Seber} or directly Levenberg or Marquardt

\section{Jacobians / Derivatives}
... show how equations can be differentiated (for later use in LMA)

\section{Confidence Region}
... of parameters
... error propagation \cite{Siegwart}
