\chapter{System Model}
\label{sec:sysmod}

Following some calculus:

\section{Motor Orientation}
We assume perfect spherical shape. That reduces the degree of freedom for the motor placement from 6 to 3. We therefore describe the motor placement by a rotation about a arbitrary axis $\mathbf{n}$
\begin{equation}
\mathbf{n} = \left(n_x, n_y, n_z\right)^\top
\end{equation}
and an rotation angle $\varphi$.
With the skew symmetric cross matrix
\begin{equation}
\textbf{a}^\times = \left[
\begin{array}{ccc}
0 & a_3 & -a_2 \\
-a_3 & 0 & a_1 \\
a_2 & -a_1 & 0
\end{array} \right]
\end{equation}
We can represent this rotation with the Gibbs-Rodriquez (XXX ???) parameters.
\begin{equation}
\boldsymbol{\lambda} = \mathbf{n}\tan{\left(\frac{\varphi}{2}\right)}
\end{equation}
Whereas this is a minimal representation (which always include singularity - here at $\varphi=\pi / 2$) we can also use non-minimal representation to overcome singularities. We therefore use the quaternion, as initially suggested by Hamilton.
\begin{equation}
\mathbf{q} = \left( \begin{array}{c} 
q_0 \\ \mathbf{q}_{1:3} 
\end{array} \right) \qquad
q_0 = \cos\left(\frac{\varphi}{2}\right) \qquad
\mathbf{q}_{1:3} = \mathbf{n}\sin\left(\frac{\varphi}{2}\right)
\end{equation}

\subsection{Rotation Matrix (or Direction Cosine Matrix)}
\begin{equation}
\mathbf{C} = \mathbf{R} = \mathbf{I} + \frac{2}{1+\boldsymbol{\lambda}^\top \boldsymbol{\lambda}}
\left(\boldsymbol{\lambda}^\times + \left(\boldsymbol{\lambda}^\times\right)^2\right)
\end{equation}


\begin{equation}
\mathbf{C} = \mathbf{R} = \mathbf{I} + 2 q_0 \mathbf{q}_{1:3}^\times + 2 \left(\mathbf{q}_{1:3}^\times\right)^2
\end{equation}
