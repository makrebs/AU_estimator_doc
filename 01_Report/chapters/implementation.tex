\chapter{Implementation}
\label{chap:implementation}

To feed the batch optimization data needs to be collected.
In this work we used both data from simulation and from a real blimp.
In this chapter we will first define what a dataset is and then 
describe the methods used to generate these datasets from a real blimp and from simulation.

\section{Dataset}
\label{sec:dataset}
The MATLAB implementation of the batch optimisation takes a dataset as the input and returns a set of optimal parameters estimated from the given dataset. \\
A dataset is designed to be self contained. 
Each dataset includes information about the structure of the system and a set of input vectors with their corresponding measurement vectors from the sensors. \\
In the current implementation the structure contains the true values for the motor positions and their coordinate system transformations, the blimp radius, the blimp mass and the inertial tensor as well as transformation matrices for the sensors. \\
The measurement data is stored in segments.
A segment is a homogeneous sequence of continuously sampled input and measurement vectors.
Because the real system exhibits transients upon changing inputs a boolean is also stored for each data-point which denotes whether the system has reached steady-state. \\
With this technique it is possible to easily subdivide the dataset into smaller sub-sets while still preserving transient information.

\section{Real System}
\label{sec:real_system}
To record data from a real blimp experiments where conducted on the Skye System. \\
The flight computer software is based upon the PIXHAWK Research Project (??cite??). 
It uses the PX4 autopilot together with QGroundControl on a laptop as the ground-station. \\
To excite the system with actuator inputs which are suitable for batch optimisation we expanded both QGroundControl and the Skye Firmware with new functionality.

\subsection{Input Patterns}
\label{sub:input_pattern}
We tried a number of different approaches to input patterns. \\
When designing input patterns it must be guaranteed that the inputs sufficiently excite the system and are applicable to the real system as well.\\
Because real world tests are conducted in a constrained environment it must be guaranteed that the blimp does not build up too translational velocities.
Additionally the rotational velocities should also be limited to prevent destruction of the system by centrifugal forces. \\
Another aspect of the real system is that it takes time until the actuators have reached the desired orientation and thrust.
The resulting transients are very hard to predict mainly because the thrust motor controller has no feedback and exhibits very high variability in its dynamic behaviour. \\
In addition to the unknown thruster dynamics the hull also reacts to load changes with vibrations that are detected by the sensors. \\
To avoid all of these problems in the optimisation it was decided to discard the data of these thrust transients. \\
By combining these two constraints on the inputs we came up with the following pattern scheme: \\
1. Apply input vector to system for a fixed amount of time\\
2. Stop thrust and turn actuators in the opposite direction\\
3. Apply same inputs in reverse to the system for the same amount of time\\
(??make proper list??)\\
Using this technique the system does not move too much while still providing good steady-state system response. \\
For the input vectors themselves we considered a few options:\\
1. use allocation to generate input vectors which result in only rotational acceleration\\
2. use fixed thrust and only change the angle of the thrusters\\
3. randomize thrust with mean and standard deviation and change the angle of the thruster\\
(??make proper list??)\\
The idea behind 1. is to further reduce the amount the blimp moves by removing the translatory forces all together. 
However this does not work. Because the allocation is designed to invert the system matrix we try to estimate the inputs and the system matrix reduce to the identity matrix the system is no long observable.\\
This leaves us with method 2. and 3.
It turns out that number 3 is better because number 2 still only excites a limited sub-space of the input space. This is avoided with the fully random inputs provided by number 3.\\

\subsection{Input Generation}
\label{sub:input_generation}
For greatest flexibility the inputs are generated on the laptop within QGroundControl and transmitted with a special direct actuator control message to the blimp. The Firmware just forwards the actuator commands to the actuators. \\
This way all of the parameters like standard deviation and time intervals can be set from within the QGroundControl interface.\\

\subsection{Testing Setup}
\label{sub:testing_setup}
Although the input patterns are designed to yield as low of a movement as possible the blimp still needs to be catched after a number of such inputs. This leads us to the basic testing setup we used to record data: \\
An operator is located at the ground station and another operator is near the blimp.
At the ground station there is a button to initiate one such input sequence.
The operator at the blimp catches it as soon as it drifts too far towards an obstacle. \\
With this setup it was possible to initiate an input sequence roughly every 6 seconds. 
This is including the time needed to reposition the blimp when it has drifted too much.

\subsection{Data Acquisition}
\label{sub:data_acquisition}
For accurate feedback from the blimp the firmware was extended to offer a mode where it will transmit (??find name for this message??) the relevant sensor and actuator feedback data to the ground station. \\
The new mode collects and transmits the following data to the ground station:\\
1. raw gyro\\
2. raw accelerometer\\
3. angular rate from EKF\\
4. angular acceleration from EKF\\
5. orientation quaternion from EKF\\
6. current thrust of each actuator\\
7. current angle of each actuator\\
(??make proper list??)\\
Because the actuator feedback loop runs at 25Hz the whole telemetry message is transmitted at that rate.
To avoid problems with noise aliasing the sensor signals are run through a resampling filter. \\
QGroundControl then writes these mavlink messages into a log file which is read by our MATLAB code.

\subsection{Data Selection}
\label{sub:data_selection}
After a raw dataset has been recorded it needs to be preprocessed to be usable for our batch optimisation code.\\
First after importing a raw dataset the regions where thrust transients are to be expected are marked in the dataset.
This process is implemented with generous margins. \\
Then the segments where the blimp has been catched and repositioned need to be cut from the dataset. \\
For this purpose a semi-automatic cutting tool was implemented. 
The tool first identifies the segments of interest and then displays each of them to the use for visual inspection.
The user can then adjust the borders of the segments to cut away any undesired disturbances. \\
Good and bad segments are easily distinguished by looking at the angular acceleration plot.
An undisturbed system shows constant angular accelerations during the periods when the thrusters are active. \\

(?? insert pic!??)

\section{Simulation}
\label{sec:simulation}
We also implemented a blimp simulator for several reasons:\\
1. fast turn around time for generating new datasets \\
2. effects like noise and drag can be easily turned on and off\\
3. arbitrary blimp configurations can be crafted within minutes\\
From this list of advantages a list of requirements can be generated:
1. include aerodynamic drag\\
2. include sensor noise\\
3. include thruster dynamics\\
4. allow for easy reconfiguration of the blimp\\
(??make proper list??)\\
These requirements led us to the following implementation:\\
(??insert pic??)
The blimp is built with objects that interact with each other.
Each of these objects can have continuous and a discrete states.\\
The simulator then interacts with the objects over an standardized interface.
At each time-step the simulator updates the states of the objects and then calls a function to calculate the derivatives of the continuous states.
Additionally, if we are at the border of a discrete time step the simulator also calls a function to get the next discrete states of the objects.\\

\subsection{Motion Equation}
\label{sub:motion_equation}
At the core of the blimp simulation is a the rigid body dynamics model. \\
The simulator uses the same motion equation as described in table~\ref{tab:sys_mod}.\\

\subsection{Mechanical Properties}
\label{sub:mech_properties}
To calculate the mechanical properties, the blimp structure is represented in a tree like data-structure. (??pictschur??)\\
The root node of the tree is the actual rigid body object which is simulated by the simulator.\\
Each node in the tree represents an assembly of parts.
With this methodology the blimp is actually an assembly of several parts. 
Normally it will include a hull, the electronics unit and a number of actuation units.
Each of these parts are assemblies by themselves.
The electronics unit could include for example the cover, a camera and the PX4 FMU, which simulates the sensors.\\
An actuation unit is also an assembly of for example a battery, a taring weight and an actuator which generates the forces and simulates the actuator dynamics.\\
This concept allows to have different types of components in a library which can then be used to build an arbitrary blimp.\\
After the blimp has been assembled the mechanical properties of the parts at the leafs of the tree data-structure are recursively combined into the next higher node. 
This way the mechanical properties of the root node can be easily recalculated.\\
The taring routine uses the tree structure to calculate values for the taring weights located at the actuation units. 
After taring is done, the changed mass of the taring weights cause the tree to recalculate the mechanical properties of the root node.

\subsection{Aerodynamic Drag}
\label{sub:aero_drag}
Aerodynamic drag is calculated in the following manner:\\
... kugel drag für hülle und alli actuatore als zylinder (formle?!) \\
zuesätzlich empirisch gefundenes drag moment vu de hülle oberfläche zum real system matche. \\

\subsection{Thruster Dynamics}
\label{sub:thrust_dynamics}
The actuator has been modelled according to test-bench measurements of the thruster.
The dynamic thruster model includes:\\
- thrust start-up delay\\
- thrust reaction delay\\
- minimum and startup thrust thresholds\\
- second order thrust motor dynamics\\
- rotation actuator maximum speed\\
- second order rotation actuator dynamics\\
Instead of trying to identify parameters with physical meanings (??ref to weichart??) a different approach is used:
The minimal parameters needed to describe the listed properties of the thruster dynamics are matched to the test-bench measurements using an empirical approach.  \\
Because the thruster dynamics are not critical to our application this yielded sufficient accuracy.

\subsection{Inertial Sensors Model}
\label{sub:imu_model}
Following the idea of the simulator framework the inertial sensors are also an object which is attached to the blimp assembly. \\
The framework of the mechanical structure goes both ways and motion of the rigid body at the root of the structure is distributed to each of the leaf nodes in their respective coordinate systems. \\
The sensor object then only has to take the already transformed accelerations, velocities and orientations and store them.
In this process the sensor object also adds noise to the sensor values.

