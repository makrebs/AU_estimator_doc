\chapter{Conclusion}
\label{chap:conclusion}

In this thesis we showed a working method to determine the actuation configuration of a spherical blimp.
On a real system like Skye, the presented method can determine the actuator configuration with a centimeter accuracy.
Additionally hard to measure parameters like the inertia tensor and the offset between COG and COB can be determined as well.
To get to this point we showed a possible parametrization of a blimp system model.
We then broke down which sensors to use and briefly examined which input vectors can be used to excite the system for identification.
We then implemented a \textsc{Matlab} framework including a modular blimp simulator to test several different optimization algorithms.
The final implementation uses Levenberg-Marquardt method for parameter optimization which showed fast and robust convergence.\\


%In this semester thesis, we have shown how parameters estimation can be applied to a multi-actuated blimp.
%We showed a possible paramterization of a blimp system model.
%Observation analysis showed which blimp properties are observable.
%Using a system model for rotational acceleration, either the scale of the motor position or the scale of the system's inertia tensor has to be known.
%Then it is possible to find an estimate of the motor configuration, the inertia tensor and the offset between the COG and COB can be found using nonlinear least squares optimization.
%A \textsc{Matlab} implementation of the Levenberg-Marquardt algorithm has been used for optimization.
%We briefly examined different input patterns.
%A sequence of random forward-backward step inputs has been found as being suitable to our application.
%Simulation and real blimp data have been generated/recorded and compared in batch optimization.

\section{Application (draft)}
There are two immediate applications for the results of this thesis.
The estimated actuation configuration can be implemented in a configuration chain that updates the allocation matrix.
The estimate of the COG offset can be used to calculate the optimal placement of the taring weights.

\paragraph{Automatic Allocation Generation}~\\
The procedure for automatic allocation generation can be realized as follows:
\begin{enumerate}
\item Prepare Skye for flight and place it in a big open space
\item Start input sequence in user interface and record feedback data
\item Run \textsc{Matlab} batch optimization algorithm
\item Upload actuation configuration data via user interface to Skye
\item Recalculate allocation matrix on firmware
\end{enumerate}
Using the recent input pattern, step 2. must be repeated iteratively and the system has to be caught when it drifts towards an object.
To get enough data for accurate configuration estimation, at least about 16 different inputs are recommended.
It takes about two minutes to record this amount of data.
This is mainly due to the transient behaviour of the trusters.
 
\paragraph{Automatic Taring Calculation}~\\
If the displacement between the center of gravity and the center of buoyancy $\mathbf{p}^{cob,cog}$ is known, the required mass for the taring weights $m^{tar^k}$ at the given positions $\mathbf{p}^{tar^k}$can be calculated to eliminate the COG offset.
This can be achieved by solving the following linear system of equations:

\begin{equation}
\left[\begin{array}{ccc}
\mathbf{p}^{tar^1} & \mathbf{p}^{tar^2} & \cdots \\
1 & 1 & \cdots
\end{array} \right]
\left[\begin{array}{c}
m^{tar^1} \\
m^{tar^2} \\
\vdots
\end{array} \right]
=
\left[\begin{array}{c}
m^{blimp} \mathbf{p}^{cob,cog} \\
m^{taring~total}
\end{array} \right]
\end{equation}

$m^{taring~total}$ can be determined by measuring the uplift of the blimp.

\section{Acknowledgement}
We would like to express our gratitude to Prof. Dr. Roland Yves Siegwart for enabling this thesis and his commitment for the Skye project.
We would also like to thank our supervisors Kostas Alexis and Markus Achtelik for their profound advice and their critical view to our ideas. \\
Further we want to express our gratitude to Dominik Werner and his team at the ETH D-BAUG Bauhalle for providing us a great hangar to fly Skye.
And finally our thanks goes to all those people who pushed the Skye project to where it is today.
Especially we thank Lukas Gasser, Miro K\"ach, Daniel Meier and Andreas Schaffner for their dedication in enabling this unique aerial platform. 


